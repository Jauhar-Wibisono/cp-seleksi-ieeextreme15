\documentclass{article}

\usepackage{geometry}
\usepackage{amsmath}
\usepackage{graphicx, eso-pic}
\usepackage{listings}
\usepackage{hyperref}
\usepackage{multicol}
\usepackage{fancyhdr}
\pagestyle{fancy}
\fancyhf{}
\hypersetup{ colorlinks=true, linkcolor=black, filecolor=magenta, urlcolor=cyan}
\geometry{ a4paper, total={170mm,257mm}, top=40mm, right=20mm, bottom=20mm, left=20mm}
\setlength{\parindent}{0pt}
\setlength{\parskip}{0.3em}
\renewcommand{\headrulewidth}{0pt}
\rfoot{\thepage}
\lfoot{Seleksi IEEEXtreme 15.0 ITB}
\lstset{
    basicstyle=\ttfamily\small,
    columns=fixed,
    extendedchars=true,
    breaklines=true,
    tabsize=2,
    prebreak=\raisebox{0ex}[0ex][0ex]{\ensuremath{\hookleftarrow}},
    frame=none,
    showtabs=false,
    showspaces=false,
    showstringspaces=false,
    prebreak={},
    keywordstyle=\color[rgb]{0.627,0.126,0.941},
    commentstyle=\color[rgb]{0.133,0.545,0.133},
    stringstyle=\color[rgb]{01,0,0},
    captionpos=t,
    escapeinside={(\%}{\%)}
}

\begin{document}

\begin{center}
    \section*{Stempel} % ganti judul soal

    \begin{tabular}{ | c c | }
        \hline
        Batas Waktu  & 1 s \\    % jangan lupa ganti time limit
        Batas Memori & 256 MB \\  % jangan lupa ganti memory limit
        \hline
    \end{tabular}
\end{center}

\subsection*{Deskripsi}
Grimstroke adalah seorang karyawan kantor pos. Salah satu tugasnya setiap hari adalah menstempel kertas-kertas. Karena sudah memiliki banyak pengalaman, Grimstroke dapat menstempel satu kertas per detik secara konsisten.

Hari ini Grimstroke melakukan tugasnya seperti biasa, menstempel $N$ buah kertas bernomor $1$ sampai $N$ secara urut membesar, namun dalam keadaan mengantuk! Setelah selesai menstempel kertas-kertasnya, Grimstroke merasa beberapa kertas distempel secara tidak benar dan ingin memeriksa pekerjaannya. Untungnya, terdapat komputer yang mencatat total kertas yang telah distempel secara benar oleh Grimstroke pada setiap detiknya. Komputer tersebut mulai mencatat \textbf{tepat setelah Grimstroke menstempel kertas pertama}. Grimstroke memberikan anda catatan komputer tersebut, bantulah dia mencari nomor-nomor kertas yang distempelnya secara tidak benar!

\subsection*{Format Masukan}
Baris pertama masukan berisi satu buah bilangan bulat $N$ $(1 \leq N \leq 100)$ - banyak kertas yang distempel Grimstroke.

Baris berikutnya berisi $N$ buah bilangan bulat $A_1, A_2, ..., A_N$ $(0 \leq A_1 \leq A_2 \leq ... \leq A_N \leq N, A_i \leq A_{i+1} \leq A_i+1$ untuk $i \epsilon [1,N-1])$. $A_i$ menyatakan total kertas yang telah distempel secara benar oleh Grimstroke sampai detik ke-$i$.


\subsection*{Format Keluaran}
Baris pertama keluaran berisi satu bilangan bulat $M$ $(0 \leq M \leq N)$ - banyak kertas yang distempel secara tidak benar oleh Grimstroke.

Baris kedua keluaran berisi $M$ bilangan bulat $B_1, B_2, ..., B_M$ $(1 \leq B_1, B_2, ..., B_M \leq N)$ - nomor-nomor kertas yang distempel secara tidak benar oleh Grimstroke.

\begin{multicols}{2}
\subsection*{Contoh Masukan 1}
\begin{lstlisting}
6
0 1 2 3 3 4
\end{lstlisting}
\null
\columnbreak
\subsection*{Contoh Keluaran 1}
\begin{lstlisting}
2
1 5
\end{lstlisting}
\vfill
\null
\end{multicols}

\begin{multicols}{2}
\subsection*{Contoh Masukan 2}
\begin{lstlisting}
1
1
\end{lstlisting}
\null
\columnbreak
\subsection*{Contoh Keluaran 2}
\begin{lstlisting}
0
\end{lstlisting}
\vfill
\null
\end{multicols}

\subsection*{Penjelasan}
\begin{itemize}
    \item Grimstroke menstempel kertas 1 secara tidak benar, sehingga $A_1 = 0$.
    \item Grimstroke menstempel kertas 2 secara benar, sehingga $A_2 = 1$.
    \item Grimstroke menstempel kertas 3 secara benar, sehingga $A_3 = 2$.
    \item Grimstroke menstempel kertas 4 secara benar, sehingga $A_4 = 3$.
    \item Grimstroke menstempel kertas 5 secara tidak benar, sehingga $A_5 = 3$.
    \item Grimstroke menstempel kertas 6 secara benar, sehingga $A_6 = 4$.
\end{itemize}
Jadi nomor-nomor kertas yang distempel secara tidak benar oleh Grimstroke adalah 1 dan 5.

\end{document}
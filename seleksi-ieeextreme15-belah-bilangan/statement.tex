\documentclass{article}

\usepackage{geometry}
\usepackage{amsmath}
\usepackage{graphicx, eso-pic}
\usepackage{listings}
\usepackage{hyperref}
\usepackage{multicol}
\usepackage{fancyhdr}
\pagestyle{fancy}
\fancyhf{}
\hypersetup{ colorlinks=true, linkcolor=black, filecolor=magenta, urlcolor=cyan}
\geometry{ a4paper, total={170mm,257mm}, top=40mm, right=20mm, bottom=20mm, left=20mm}
\setlength{\parindent}{0pt}
\setlength{\parskip}{0.3em}
\renewcommand{\headrulewidth}{0pt}
\rfoot{\thepage}
\lfoot{Seleksi IEEEXtreme 15.0 ITB}
\lstset{
    basicstyle=\ttfamily\small,
    columns=fixed,
    extendedchars=true,
    breaklines=true,
    tabsize=2,
    prebreak=\raisebox{0ex}[0ex][0ex]{\ensuremath{\hookleftarrow}},
    frame=none,
    showtabs=false,
    showspaces=false,
    showstringspaces=false,
    prebreak={},
    keywordstyle=\color[rgb]{0.627,0.126,0.941},
    commentstyle=\color[rgb]{0.133,0.545,0.133},
    stringstyle=\color[rgb]{01,0,0},
    captionpos=t,
    escapeinside={(\%}{\%)}
}

\begin{document}

\begin{center}
    \section*{Belah Bilangan} % ganti judul soal

    \begin{tabular}{ | c c | }
        \hline
        Batas Waktu  & 1 s \\    % jangan lupa ganti time limit
        Batas Memori & 256 MB \\  % jangan lupa ganti memory limit
        \hline
    \end{tabular}
\end{center}

\subsection*{Deskripsi}
Diberikan bilangan bulat non-negatif $A$. Anda dapat melakukan operasi berikut berkali-kali (mungkin nol kali).

\begin{itemize}
    \item Ubah semua digit $d$ di $A$ menjadi $\lfloor \frac{d}{2} \rfloor$, lalu hapus semua \href{https://en.wikipedia.org/wiki/Leading_zero}{\textit{leading zero}} dari $A$. Khususnya, apabila semua digit $A$ merupakan nol, nilai $A$ menjadi nol setelah semua \textit{leading zero}-nya dihapus.
\end{itemize}

Berapa operasi minimal yang dibutuhkan untuk mengubah $A$ menjadi nol?

\subsection*{Format Masukan}
Masukan terdiri atas satu baris berisi bilangan bulat non-negatif $A$ $(0 \leq A \leq 10^{100})$.

\subsection*{Format Keluaran}
Keluaran terdiri atas satu baris berisi bilangan yang menyatakan banyak operasi minimal yang dibutuhkan untuk mengubah $A$ menjadi nol.

\begin{multicols}{2}
\subsection*{Contoh Masukan 1}
\begin{lstlisting}
1010
\end{lstlisting}
\null
\columnbreak
\subsection*{Contoh Keluaran 1}
\begin{lstlisting}
1
\end{lstlisting}
\vfill
\null
\end{multicols}

\begin{multicols}{2}
\subsection*{Contoh Masukan 2}
\begin{lstlisting}
1337
\end{lstlisting}
\null
\columnbreak
\subsection*{Contoh Keluaran 2}
\begin{lstlisting}
3
\end{lstlisting}
\vfill
\null
\end{multicols}

\begin{multicols}{2}
\subsection*{Contoh Masukan 3}
\begin{lstlisting}
0
\end{lstlisting}
\null
\columnbreak
\subsection*{Contoh Keluaran 3}
\begin{lstlisting}
0
\end{lstlisting}
\vfill
\null
\end{multicols}

\subsection*{Penjelasan}
\begin{itemize}
    \item Pada contoh 1, perubahan yang dialami $A$ adalah $1010 \longrightarrow 0$. Terlihat bahwa dibutuhkan satu operasi untuk mengubah $A$ menjadi nol.
    \item Pada contoh 2, perubahan yang dialami $A$ adalah $1337 \longrightarrow 113 \longrightarrow 1 \longrightarrow 0$. Terlihat bahwa dibutuhkan tiga operasi untuk mengubah $A$ menjadi nol.
    \item Pada contoh 3, $A$ sudah bernilai nol, sehingga tidak dibutuhkan operasi untuk mengubah $A$ menjadi nol.
\end{itemize}

\end{document}
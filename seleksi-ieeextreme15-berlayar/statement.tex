\documentclass{article}

\usepackage{geometry}
\usepackage{amsmath}
\usepackage{graphicx, eso-pic}
\usepackage{listings}
\usepackage{hyperref}
\usepackage{multicol}
\usepackage{fancyhdr}
\pagestyle{fancy}
\fancyhf{}
\hypersetup{ colorlinks=true, linkcolor=black, filecolor=magenta, urlcolor=cyan}
\geometry{ a4paper, total={170mm,257mm}, top=40mm, right=20mm, bottom=20mm, left=20mm}
\setlength{\parindent}{0pt}
\setlength{\parskip}{0.3em}
\renewcommand{\headrulewidth}{0pt}
\rfoot{\thepage}
\lfoot{Seleksi IEEEXtreme 15.0 ITB}
\lstset{
    basicstyle=\ttfamily\small,
    columns=fixed,
    extendedchars=true,
    breaklines=true,
    tabsize=2,
    prebreak=\raisebox{0ex}[0ex][0ex]{\ensuremath{\hookleftarrow}},
    frame=none,
    showtabs=false,
    showspaces=false,
    showstringspaces=false,
    prebreak={},
    keywordstyle=\color[rgb]{0.627,0.126,0.941},
    commentstyle=\color[rgb]{0.133,0.545,0.133},
    stringstyle=\color[rgb]{01,0,0},
    captionpos=t,
    escapeinside={(\%}{\%)}
}

\begin{document}

\begin{center}
    \section*{Berlayar} % ganti judul soal

    \begin{tabular}{ | c c | }
        \hline
        Batas Waktu  & 1 s \\    % jangan lupa ganti time limit
        Batas Memori & 256 MB \\  % jangan lupa ganti memory limit
        \hline
    \end{tabular}
\end{center}

\subsection*{Deskripsi}
Di kepulauan Claddish terdapat $N+1$ pulau yang berjejeran dari barat ke timur dan dinomori $0$ sampai $N$. Setiap pulau memiliki $N$ kapal, dinomori $1$ sampai $N$. Kapal $i$ di pulau $j$ akan membawa penumpang dari pulau $j$ ke pulau $(j+i)$ (apabila pulau tersebut ada).

Kunkka akan menempuh perjalanan dari pulau $0$ dengan cara berikut.

\begin{itemize}
    \item Pertama, Kunkka akan menaiki kapal $K$ dari pulau $0$.
    \item Lalu, ia akan melanjutkan perjalanannya dengan aturan berikut. Misalkan Kunkka menaiki kapal bernomor $p$ untuk mencapai pulau tempatnya sekarang dari pulau sebelumnya. Untuk mencapai pulau selanjutnya, Kunkka akan menaiki kapal bernomor $(p-1)$, $p$, atau $(p+1)$. Perhatikan bahwa Kunkka tidak dapat memilih nomor kapal yang tidak ada atau menaiki kapal ke pulau yang tidak ada.
\end{itemize}

Didefinisikan rute kapal sebagai sekuens nomor kapal yang dinaiki Kunkka dengan cara di atas. Ada berapa banyak rute kapal berbeda yang membawa Kunkka ke pulau $N$? Jawablah dalam modulo $10^9+7$!

Dua rute kapal $S$ dan $T$ dianggap berbeda apabila salah satu dari dua syarat berikut dipenuhi.

\begin{itemize}
    \item Banyak kapal di $S$ dan $T$ berbeda.
    \item Terdapat bilangan bulat positif $i$ sehingga $S_i \neq T_i$ (dengan kata lain, kapal ke-$i$ yang diambil di rute $S$ dan rute $R$ berbeda).
\end{itemize}

\subsection*{Format Masukan}
Masukan terdiri atas satu baris berisi dua bilangan bulat $N$ dan $K$ $(1 \leq K \leq N \leq 3 \times 10^4)$. $N$ menyatakan banyak pulau (dikurangi satu) dan banyak kapal di setiap pulau, sedangkan $K$ menyatakan kapal pertama yang dinaiki Kunkka.

\subsection*{Format Keluaran}
Keluaran terdiri atas satu baris berisi bilangan bulat yang menyatakan banyak rute kapal berbeda yang membawa Kunkka ke pulau $N$ dimodulo $10^9+7$.

\begin{multicols}{2}
\subsection*{Contoh Masukan 1}
\begin{lstlisting}
5 2
\end{lstlisting}
\null
\columnbreak
\subsection*{Contoh Keluaran 1}
\begin{lstlisting}
4
\end{lstlisting}
\vfill
\null
\end{multicols}

\begin{multicols}{2}
\subsection*{Contoh Masukan 2}
\begin{lstlisting}
5 4
\end{lstlisting}
\null
\columnbreak
\subsection*{Contoh Keluaran 2}
\begin{lstlisting}
0
\end{lstlisting}
\vfill
\null
\end{multicols}

\begin{multicols}{2}
\subsection*{Contoh Masukan 2}
\begin{lstlisting}
21487 4
\end{lstlisting}
\null
\columnbreak
\subsection*{Contoh Keluaran 2}
\begin{lstlisting}
631858925
\end{lstlisting}
\vfill
\null
\end{multicols}

\subsection*{Penjelasan}
Rute-rute kapal yang dapat diambil Kunkka pada contoh 1 adalah sebagai berikut.
\begin{itemize}
    \item $[2, 3]$
    \item $[2, 2, 1]$
    \item $[2, 1, 2]$
    \item $[2, 1, 1, 1]$
\end{itemize}
Pada contoh 2, Kunkka hanya dapat menaiki kapal bernomor $3$, $4$, dan $5$ dari pulau $4$. Di antara kapal yang bisa dinaikinya, tidak ada kapal yang membawanya ke pulau 5, sehingga jawaban contoh 2 adalah $0$.

\end{document}
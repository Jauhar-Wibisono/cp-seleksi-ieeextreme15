\documentclass{article}

\usepackage{geometry}
\usepackage{amsmath}
\usepackage{graphicx, eso-pic}
\usepackage{listings}
\usepackage{hyperref}
\usepackage{multicol}
\usepackage{fancyhdr}
\pagestyle{fancy}
\fancyhf{}
\hypersetup{ colorlinks=true, linkcolor=black, filecolor=magenta, urlcolor=cyan}
\geometry{ a4paper, total={170mm,257mm}, top=40mm, right=20mm, bottom=20mm, left=20mm}
\setlength{\parindent}{0pt}
\setlength{\parskip}{0.3em}
\renewcommand{\headrulewidth}{0pt}
\rfoot{\thepage}
\lfoot{Seleksi IEEEXtreme 15.0 ITB}
\lstset{
    basicstyle=\ttfamily\small,
    columns=fixed,
    extendedchars=true,
    breaklines=true,
    tabsize=2,
    prebreak=\raisebox{0ex}[0ex][0ex]{\ensuremath{\hookleftarrow}},
    frame=none,
    showtabs=false,
    showspaces=false,
    showstringspaces=false,
    prebreak={},
    keywordstyle=\color[rgb]{0.627,0.126,0.941},
    commentstyle=\color[rgb]{0.133,0.545,0.133},
    stringstyle=\color[rgb]{01,0,0},
    captionpos=t,
    escapeinside={(\%}{\%)}
}

\begin{document}

\begin{center}
    \section*{Piramid} % ganti judul soal

    \begin{tabular}{ | c c | }
        \hline
        Batas Waktu  & 1 s \\    % jangan lupa ganti time limit
        Batas Memori & 128 MB \\  % jangan lupa ganti memory limit
        \hline
    \end{tabular}
\end{center}

\subsection*{Deskripsi}
Pada kerajan Mesin Kuno, Raja Mesir Seth memerintahkan penyihir lokal/shaman di sekitarnya dan menemukan kandidat kuat bernama Rhasta. Rhasta diminta raja untuk membangun piramida 2 dimensi, yang terdiri dari sejumlah balok berukuran sama. Raja memberikan $N$ balok pada Rhasta, yang tiap batunya memiliki nilai berat (tidak mesti berbeda). $N$ balok tersebut harus diletakkan di dasar piramida. Setiap balok piramida di atas lapisan dasar harus bertumpu pada tepat 2 balok dibawahnya, serta memiliki berat senilai jumlah berat 2 balok yang ditumpukan tersebut. Tentukan berat total piramida terkecil yang dapat dibangun Rhasta dengan menggunakan semua balok yang diberikan Raja dalam modulo $1000000007$!

\subsection*{Format Masukan}

Baris pertama berisi satu bilangan bulat $N$ $(1 \leq N \leq 2\times 10^5)$.

Baris kedua berisi $N$ bilangan bulat $a_1, a_2, \cdots, a_N$ yang memenuhi $1 \leq a_i \leq 10^9$.


\subsection*{Format Keluaran}
Keluarkan satu bilangan $M$ yang senilai dengan berat total semua balok pada piramida terkecil yang mungkin dalam modulo $1000000007$.

\begin{multicols}{2}
\subsection*{Contoh Masukan 1}
\begin{lstlisting}
4
1 2 3 4
\end{lstlisting}
\null
\columnbreak
\subsection*{Contoh Keluaran 1}
\begin{lstlisting}

55

\end{lstlisting}
\vfill
\null
\end{multicols}

\subsection*{Penjelasan}
Berat total terkecil dapat dicapai oleh Rhasta dengan susunan:
\begin{lstlisting}
  16
  8 8
 5 3 5
4 1 2 3
\end{lstlisting}

\end{document}
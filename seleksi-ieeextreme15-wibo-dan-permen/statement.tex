\documentclass{article}

\usepackage{geometry}
\usepackage{amsmath}
\usepackage{graphicx, eso-pic}
\usepackage{listings}
\usepackage{hyperref}
\usepackage{multicol}
\usepackage{fancyhdr}
\pagestyle{fancy}
\fancyhf{}
\hypersetup{ colorlinks=true, linkcolor=black, filecolor=magenta, urlcolor=cyan}
\geometry{ a4paper, total={170mm,257mm}, top=40mm, right=20mm, bottom=20mm, left=20mm}
\setlength{\parindent}{0pt}
\setlength{\parskip}{0.3em}
\renewcommand{\headrulewidth}{0pt}
\rfoot{\thepage}
\lfoot{Seleksi IEEEXtreme 15.0 ITB}
\lstset{
    basicstyle=\ttfamily\small,
    columns=fixed,
    extendedchars=true,
    breaklines=true,
    tabsize=2,
    prebreak=\raisebox{0ex}[0ex][0ex]{\ensuremath{\hookleftarrow}},
    frame=none,
    showtabs=false,
    showspaces=false,
    showstringspaces=false,
    prebreak={},
    keywordstyle=\color[rgb]{0.627,0.126,0.941},
    commentstyle=\color[rgb]{0.133,0.545,0.133},
    stringstyle=\color[rgb]{01,0,0},
    captionpos=t,
    escapeinside={(\%}{\%)}
}

\begin{document}

\begin{center}
    \section*{Kue} % ganti judul soal

    \begin{tabular}{ | c c | }
        \hline
        Batas Waktu  & 1 s \\    % jangan lupa ganti time limit
        Batas Memori & 256 MB \\  % jangan lupa ganti memory limit
        \hline
    \end{tabular}
\end{center}

\subsection*{Deskripsi}
Mortimer ingin membelikan kue untuk $M$ orang muridnya. Karena belum memiliki kue, ia pergi ke toko kue Beatrix.

Di toko kue Beatrix terdapat $N$ bungkusan kue yang dinomori $1$ sampai $N$. Bungkusan ke-$i$ berisi $A_i$ kue. Agar adil, Mortimer ingin memilih beberapa bungkusan sehingga total jumlah kue di dalam bungkusan-bungkusan tersebut dapat dibagi habis oleh jumlah muridnya, yaitu $M$. Bantulah Mortimer mencari satu kemungkinan kumpulan bungkusan yang dapat dipilihnya, atau nyatakan bahwa tidak ada kumpulan bungkusan yang dapat dipilihnya!

\subsection*{Format Masukan}
Baris pertama masukan berisi dua bilangan bulat $N$ dan $M$ $(1 \leq M < N \leq 2\times10^5)$ - banyak bungkusan kue dan banyak murid Mortimer. 

Baris berikutnya berisi $N$ buah bilangan bulat $A_1, A_2, ..., A_N$ $(1 \leq A_1, A_2, ..., A_N \leq 10^9)$ - banyak kue di bungkusan-bungkusan.

\subsection*{Format Keluaran}
Apabila tidak ada kumpulan bungkusan dengan total jumlah kue yang dapat dibagi habis oleh jumlah murid Mortimer, keluaran berisi satu bilangan $-1$.

Apabila sebaliknya, baris pertama keluaran berisi satu bilangan bulat $K$ $(1 \leq K \leq N)$ yang menyatakan banyak kantong di kumpulan kantong yang dapat dipilih Mortimer. Baris berikutnya berisi $K$ bilangan bulat yang menyatakan nomor-nomor kantong di kumpulan kantong. Apabila terdapat banyak kemungkinan jawaban, \textbf{keluarkan yang mana saja}.

\begin{multicols}{2}
\subsection*{Contoh Masukan 1}
\begin{lstlisting}
5 4
1 2 3 7 5
\end{lstlisting}
\columnbreak

\subsection*{Contoh Keluaran 1}
\begin{lstlisting}
2
1 4
\end{lstlisting}
\vfill
\null
\end{multicols}

\begin{multicols}{2}
\subsection*{Contoh Masukan 2}
\begin{lstlisting}
5 3
2 2 2 2 2
\end{lstlisting}
\columnbreak

\subsection*{Contoh Keluaran 2}
\begin{lstlisting}
3
1 3 4
\end{lstlisting}
\vfill
\null
\end{multicols}

\begin{multicols}{2}
\subsection*{Contoh Masukan 3}
\begin{lstlisting}
3 2
8 12 16
\end{lstlisting}
\columnbreak

\subsection*{Contoh Keluaran 3}
\begin{lstlisting}
1
1
\end{lstlisting}
\vfill
\null
\end{multicols}

\subsection*{Penjelasan}
Pada contoh 1, total jumlah kue adalah $A_1 + A_4 = 1 + 7 = 8$ sehingga dapat dibagi habis oleh jumlah murid Mortimer, yaitu $4$.

\end{document}
\documentclass{article}

\usepackage{geometry}
\usepackage{amsmath}
\usepackage{graphicx, eso-pic}
\usepackage{listings}
\usepackage{hyperref}
\usepackage{multicol}
\usepackage{fancyhdr}
\pagestyle{fancy}
\fancyhf{}
\hypersetup{ colorlinks=true, linkcolor=black, filecolor=magenta, urlcolor=cyan}
\geometry{ a4paper, total={170mm,257mm}, top=40mm, right=20mm, bottom=20mm, left=20mm}
\setlength{\parindent}{0pt}
\setlength{\parskip}{0.3em}
\renewcommand{\headrulewidth}{0pt}
\rfoot{\thepage}
\lfoot{Seleksi IEEEXtreme 15.0 ITB}
\lstset{
    basicstyle=\ttfamily\small,
    columns=fixed,
    extendedchars=true,
    breaklines=true,
    tabsize=2,
    prebreak=\raisebox{0ex}[0ex][0ex]{\ensuremath{\hookleftarrow}},
    frame=none,
    showtabs=false,
    showspaces=false,
    showstringspaces=false,
    prebreak={},
    keywordstyle=\color[rgb]{0.627,0.126,0.941},
    commentstyle=\color[rgb]{0.133,0.545,0.133},
    stringstyle=\color[rgb]{01,0,0},
    captionpos=t,
    escapeinside={(\%}{\%)}
}

\begin{document}

\begin{center}
    \section*{Pertanyaan Ganjil Genap} % ganti judul soal

    \begin{tabular}{ | c c | }
        \hline
        Batas Waktu  & 1 s \\    % jangan lupa ganti time limit
        Batas Memori & 256 MB \\  % jangan lupa ganti memory limit
        \hline
    \end{tabular}
\end{center}

\subsection*{Deskripsi}
Terdapat sebuah matriks dua dimensi $A$ berisi bilangan bulat dan berukuran $N \times N$. Baris-baris \textit{grid} ini dinomori $1$ sampai $N$ dari atas ke bawah, sedangkan kolom-kolom dinomori $1$ sampai $N$ dari kiri ke kanan. Pada awalnya, setiap sel di $A$ bernilai nol. Terdapat tiga jenis operasi yang bisa dilakukan terhadap A.
\begin{itemize}
    \item $1\ B\ X$: menambahkan bilangan bulat $X$ ke sel-sel di baris $B$
    \item $2\ K\ X$: menambahkan bilangan bulat $X$ ke sel-sel di kolom $K$
    \item $3\ B1\ K1\ B2\ K2$: menanyakan "Apakah jumlah nilai di submatriks dari baris $B1$ kolom $K1$ hingga baris $B2$ kolom $K2$ dari $A$ ganjil?"
\end{itemize}
Anda diminta untuk melakukan Q buah operasi.

\subsection*{Format Masukan}
Baris pertama masukan berisi dua bilangan bulat $N$ dan $Q$ $(1 \leq N, Q \leq 10^5)$ - panjang sisi matriks dan banyak operasi.

Masukan lalu dilanjutkan dengan $Q$ baris. Di awal baris ke-$i$ terdapat bilangan bulat $T_i$ $(1 \leq T_i \leq 3)$ - jenis operasi. Setelah itu, baris masukan dilanjutkan seperti berikut.
\begin{itemize}
    \item Apabila $T_i = 1$, masukan dilanjutkan dengan dua bilangan bulat $B_i$ dan $X_i$ $(1 \leq B_i \leq N, 1 \leq X_i \leq 10^9)$ - nomor baris dan nilai tambahan.
    \item Apabila $T_i = 2$, masukan dilanjutkan dengan dua bilangan bulat $K_i$ dan $X_i$ $(1 \leq K_i \leq N, 1 \leq X_i \leq 10^9)$ - nomor kolom dan nilai tambahan.
    \item Apabila $T_i = 3$, masukan dilanjutkan dengan empat bilangan bulat $B1_{i}$ $K1_{i}$ $B2_{i}$ dan $K2_{i}$ $(1 \leq B1_i \leq B2_i \leq N, 1 \leq K1_i \leq K2_i \leq N)$ - nomor baris dan kolom sel kiri atas dan nomor baris dan kolom sel kanan bawah.
\end{itemize}
Dijamin terdapat setidaknya satu operasi jenis $3$.

\subsection*{Format Keluaran}
Untuk setiap operasi jenis 3, keluarkan \verb|Ganjil| apabila jumlah nilai di submatriks ganjil dan \verb|Genap| apabila jumlah nilai di submatriks genap.

\begin{multicols}{2}
\subsection*{Contoh Masukan}
\begin{lstlisting}
5 5
1 4 1
3 3 3 5 5
2 4 3
3 1 2 4 5
3 1 4 1 4
\end{lstlisting}
\null
\columnbreak
\subsection*{Contoh Keluaran}
\begin{lstlisting}
Ganjil
Genap
Ganjil
\end{lstlisting}
\vfill
\null
\end{multicols}

\end{document}
\documentclass{article}

\usepackage{geometry}
\usepackage{amsmath}
\usepackage{graphicx, eso-pic}
\usepackage{listings}
\usepackage{hyperref}
\usepackage{multicol}
\usepackage{fancyhdr}
\pagestyle{fancy}
\fancyhf{}
\hypersetup{ colorlinks=true, linkcolor=black, filecolor=magenta, urlcolor=cyan}
\geometry{ a4paper, total={170mm,257mm}, top=40mm, right=20mm, bottom=20mm, left=20mm}
\setlength{\parindent}{0pt}
\setlength{\parskip}{0.3em}
\renewcommand{\headrulewidth}{0pt}
\rfoot{\thepage}
\lfoot{Seleksi IEEEXtreme 15.0 ITB}
\lstset{
    basicstyle=\ttfamily\small,
    columns=fixed,
    extendedchars=true,
    breaklines=true,
    tabsize=2,
    prebreak=\raisebox{0ex}[0ex][0ex]{\ensuremath{\hookleftarrow}},
    frame=none,
    showtabs=false,
    showspaces=false,
    showstringspaces=false,
    prebreak={},
    keywordstyle=\color[rgb]{0.627,0.126,0.941},
    commentstyle=\color[rgb]{0.133,0.545,0.133},
    stringstyle=\color[rgb]{01,0,0},
    captionpos=t,
    escapeinside={(\%}{\%)}
}

\begin{document}

\begin{center}
    \section*{Closest Cell} % ganti judul soal

    \begin{tabular}{ | c c | }
        \hline
        Batas Waktu  & 1s \\    % jangan lupa ganti time limit
        Batas Memori & 256MB \\  % jangan lupa ganti memory limit
        \hline
    \end{tabular}
\end{center}

\subsection*{Deskripsi}
Pada update 7.37, Wisp mendapatkan buff pada skill pertamanya, "Tether"!

Singkat cerita, Wisp adalah sebuah objek yang mendapatkan kekuatan dari teman-temanya melalui tether ini. Pada update ini, Wisp mendapatkan kekuatan semakin besar seiring bertambah jauhnya jarak dia dari teman yang di-tether. Asumsikan game berletak pada bidang 2 dimensi dan tiap objek (Wisp dan $N$ temannya) berada pada koordinat $(x, y)$ untuk suatu $x$ dan $y$ bilangan bulat yang memenuhi $-10^5 \leq x, y \leq 10^5$.

Wisp selalu ter-tether dengan setiap temannya (total $N$ tether). Developer dari game ini menginginkan setiap tether dengan teman $t_i$ menambah kekuatan Wisp sebesar kuadrat dari jarak euclidian Wisp dengan $t_i$. Namun karena bug, tiap teman $t_i$ memiliki 2 bilangan $a_i$ dan $b_i$ sehingga jika $t_i$ berkoordinat $(x_i, y_i)$, dan Wisp berkoordinat $(x, y)$, maka dari tether tersebut kekuatan Wisp bertambah sebesar
$$ max((x_i - x)^2, a_i) + max((y_i - y)^2, b_i)$$
Tentukan \textbf{total kekuatan minimal} yang dapat diperoleh dalam peletakan Wisp pada bidang jika diberikan $N$ quadruplet nilai $x, y, a, b$ untuk tiap temannya

\subsection*{Format Masukan}

Baris pertama berisi bilangan bulat $1 \leq N \leq 10^5$.

$N$ baris berikutnya berisi 4 bilangan bulat $-10^5 \leq x_i, y_i \leq 10^5$, $0 \leq a_i, b_i \leq 10^9$.


\subsection*{Format Keluaran}
Baris pertama berisi total kekuatan minimal yang dapat dimiliki Wisp.

\begin{multicols}{2}
\subsection*{Contoh Masukan 1}
\begin{lstlisting}
4
1 4 0 0
3 2 0 0
4 3 0 4
2 1 4 0
\end{lstlisting}
\null
\columnbreak
\subsection*{Contoh Keluaran 1}
\begin{lstlisting}
18
\end{lstlisting}
\vfill
\null
\end{multicols}


\subsection*{Penjelasan}
Pada contoh masukan, Wisp dapat diletakkan pada koordinat $(3, 2)$ dengan total kekuatan dari masing-masing teman $8 + 0 + 5 + 5 = 18$

\end{document}